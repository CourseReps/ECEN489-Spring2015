\documentclass[11pt]{article}

%%  Dimensions and URL
\usepackage[margin=1in]{geometry}
\usepackage{hyperref}

%%  Definitions
\renewcommand{\baselinestretch}{1.1}
\pagestyle{plain}


\begin{document}

\begin{center}
{\bfseries \LARGE Task 3 -- Data Management}
\end{center}


\section{SQLite}

SQLite is a software library that implements a self-contained, serverless, SQL database engine.
It is used extensively, and it ships with the Android operating system.
Furthermore, the source code for SQLite is in the public domain.
\begin{center}
\url{http://www.sqlite.org}
\end{center}
Unlike most SQL databases, SQLite does not require a separate server process; it reads and writes directly to disk files.
There is a \texttt{C/C++} interface to SQLite, namely \texttt{sqlite3}.


\subsection*{Action Items}

\begin{itemize}
\item \textbf{Download and Install:} SQLite. \\
\url{http://www.sqlite.org/download.html}
\item \textbf{Read:} The SQLite Quick Start Guide.\\
\url{http://www.sqlite.org/quickstart.html}\\
\url{http://www.sqlite.org/sqlite.html}
\end{itemize}


\subsection*{Proficient Test}

\begin{enumerate}
\item Using command-line tools, create a new database.
Add and edit entries.
Search and display the content of your database.
\item Employ a GUI to display the content of the database.\\
\url{https://addons.mozilla.org/en-US/firefox/addon/sqlite-manager/}
\item (Optional) Repeat Step~1 on an Android phone using \texttt{SQLiteOpenHelper} and \texttt{SQLiteDatabase}.\\
\texttt{import android.database.sqlite.SQLiteDatabase;}\\
\texttt{import android.database.sqlite.SQLiteOpenHelper;}
\end{enumerate}


\end{document}

