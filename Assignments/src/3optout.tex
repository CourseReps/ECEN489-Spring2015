\documentclass[11pt]{article}

%%  Dimensions and URL
\usepackage[margin=1in]{geometry}
\usepackage{hyperref}

%%  Definitions
\renewcommand{\baselinestretch}{1.1}
\pagestyle{plain}


\begin{document}

\begin{center}
{\bfseries \LARGE Opt-Out Challenge 3}
\end{center}

\paragraph{Reading Assignment:}
How to Program Java, 10th edition
\begin{itemize}
\item Chapter~7 -- Arrays and ArrayList
\item Chapter~8 -- Classes and Objects: A Deeper Look
\item Chapter~9 -- Object-Oriented Programming: Inheritance
\end{itemize}


\section*{Programming Challenges}

Create a server-client infrastructure where a client connects to the server and periodically report the status of local parameters such as time, Wi-Fi connectivity level, mouse position.
Upon accepting a connection, the server should create a table for the remote computer using \href{http://www.sqlite.org/}{SQLite}.
The server should subsequently log the parameters as they are reported by the client.
The client should be able to access the server over the Internet.
You may want to leverage the Java Database Connectivity (JDBC) API in your implementation.


\subsection*{Code}

\begin{enumerate}
\item Implement the server and the client in Java.
\item Using IntelliJ IDEA, Git, and GitHub, commit your code for the server as a project labeled \texttt{Challenge3server} under \texttt{Students/<GitHubID>/}, where \texttt{<GitHubID>} should be replaced by your username on \href{https://GitHub.com}{GitHub}.
\item In a similar fashion, commit your code for the client as a project labeled \texttt{Challenge3client} under \texttt{Students/<GitHubID>/}.
\item As part of your demonstration, you will have to use an \href{http://www.sqlite.org/}{SQLite} visualizer to showcase the content of the table before and after the connection.
\end{enumerate}


\end{document}
