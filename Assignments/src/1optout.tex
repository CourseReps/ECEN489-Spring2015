\documentclass[letterpaper,11pt]{article}

%%  Dimensions and URL
\usepackage[margin=1in]{geometry}
\usepackage{hyperref}

%%  Definitions
\renewcommand{\baselinestretch}{1.1}
\pagestyle{plain}


\begin{document}

\begin{center}
{\bfseries \LARGE Opt-Out Assignment 1}
\end{center}

\noindent
\rule[1mm]{\linewidth}{0.2pt}

\section*{Programming Challenges}

Create a \texttt{GatheredData} class that contains the following fields, along with appropriate methods.
\begin{itemize}
\item \texttt{time (long)}
\item \texttt{longitude (double)}
\item \texttt{latitude (double)}
\end{itemize}
Write a program that takes two objects (instances) of this class and computes the average velocity between the two points, assuming motion along the shortest path.
If you are not familiar with the haversine formula, you may want to look it up before you implement your program.


\end{document}
